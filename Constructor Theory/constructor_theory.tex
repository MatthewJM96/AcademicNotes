%%%%%%%%%%%%%%%%%%%%%%%%%%%%%%%%%%%%%%%%%%%%%%%%%%%
%% LaTeX book template                           %%
%% Author:  Amber Jain (http://amberj.devio.us/) %%
%% License: ISC license                          %%
%%%%%%%%%%%%%%%%%%%%%%%%%%%%%%%%%%%%%%%%%%%%%%%%%%%

\documentclass[11pt]{report}
%%%%%%%%%%%%%%%%%%%%%%%%%%%%%%%%%%%%%%%%%%%%%%%%%%%%%%%%%
% Source: http://en.wikibooks.org/wiki/LaTeX/Hyperlinks %
%%%%%%%%%%%%%%%%%%%%%%%%%%%%%%%%%%%%%%%%%%%%%%%%%%%%%%%%%
\usepackage{hyperref}
\usepackage{graphicx}
\usepackage[english]{babel}


\usepackage{slashed}
\usepackage{amsmath}
\usepackage{amssymb}
\usepackage{color}
\usepackage{chngpage}

\graphicspath{ {Images/} }

\renewcommand{\arraystretch}{1.3}

\usepackage{mathtools}

\usepackage{bbm}
\usepackage{urwchancal}
\usepackage{pifont}

\newcommand{\cmark}{\ding{51}}%
\newcommand{\xmark}{\ding{55}}%


\usepackage{xparse}
%
\DeclarePairedDelimiterX{\set}[1]{\{}{\}}{\setargs{#1}}
\NewDocumentCommand{\setargs}{>{\SplitArgument{1}{;}}m}
{\setargsaux#1}
\NewDocumentCommand{\setargsaux}{mm}
{\IfNoValueTF{#2}{#1} {#1\,\delimsize|\,\mathopen{}#2}}%{#1\:;\:#2}

\DeclarePairedDelimiter\abs{\lvert}{\rvert}%
\DeclarePairedDelimiter\norm{\lVert}{\rVert}%

% Swap the definition of \abs* and \norm*, so that \abs
% and \norm resizes the size of the brackets, and the 
% starred version does not.
\makeatletter
\let\oldabs\abs
\def\abs{\@ifstar{\oldabs}{\oldabs*}}
%
\let\oldnorm\norm
\def\norm{\@ifstar{\oldnorm}{\oldnorm*}}
\makeatother

%%%%%%%%%%%%%%%%%%%%%%%%%%%%%%%%%%%%%%%%%%%%%%%%
% Chapter quote at the start of chapter        %
% Source: http://tex.stackexchange.com/a/53380 %
%%%%%%%%%%%%%%%%%%%%%%%%%%%%%%%%%%%%%%%%%%%%%%%%
\makeatletter
\renewcommand{\@chapapp}{}% Not necessary...
\newenvironment{chapquote}[2][2em]
  {\setlength{\@tempdima}{#1}%
   \def\chapquote@author{#2}%
   \parshape 1 \@tempdima \dimexpr\textwidth-2\@tempdima\relax%
   \itshape}
  {\par\normalfont\hfill--\ \chapquote@author\hspace*{\@tempdima}\par\bigskip}
\makeatother

%%%%%%%%%%%%%%%%%%%%%%%%%%%%%%%%%%%%%%%%%%%%%%%%%%%
% First page of book which contains 'stuff' like: %
%  - Book title, subtitle                         %
%  - Book author name                             %
%%%%%%%%%%%%%%%%%%%%%%%%%%%%%%%%%%%%%%%%%%%%%%%%%%%

% Book's title and subtitle
\title{\Huge \textbf{Constructor Theory} \\ \Large Notes on a Revolution.\thanks{credit to David Deutsch, Chiara Marletto, and colleagues}}
% Author
\author{\textsc{by Matthew Marshall}}


\begin{document}

\maketitle

%%%%%%%%%%%%%%%%%%%%%%%%%%%%%%%%%%%%%%%%%%%%%%%%%%%%%%%%%%%%%%%%%%%%%%%%
% Auto-generated table of contents, list of figures and list of tables %
%%%%%%%%%%%%%%%%%%%%%%%%%%%%%%%%%%%%%%%%%%%%%%%%%%%%%%%%%%%%%%%%%%%%%%%%
\tableofcontents

%%%%%%%%%%%
% Preface %
%%%%%%%%%%%
\chapter*{Preface}
These notes are likely incomplete and inaccurate in places, feel free to email me\footnote{\url{matthew.marshall@stfc.ac.uk}} and I will endeavour to make corrections.

\section*{The Why}
Since physics developed out of natural philosophy, it has presumed a dogma that might be now called the Newtonian framing: identify a rule of how the state of a thing changes into another state, and figure out some fixing of that thing's state at some "time" - concisely: laws of motion and initial conditions. From these two things, with arbitrary precision and (until quantum theory) accuracy we can say how that thing has been throughout its being. Quantum mechanics introduced an uncertainty to this, but that uncertainty may well be a local property of where we are (a modal branch of the universe, or some hidden variables that we have yet to, or cannot determine), and even if not the basic way of thinking about systems in still largely the same.

Aside from these laws of motion and initial conditions, we have had other kinds of laws which simply state certain things are impossible (e.g. conservations/symmetries, 2nd law of thermodynamics, etc.), which in turn give us limits on how efficiently certain processes can occur (e.g. heat exchange). This project seeks to reformulate physics (and perhaps the other sciences) in terms of statements of possible and impossible transformations - with possibility and impossibility used in a binary sense.

To accommodate such a reformulation, constructor theory seeks to provide a framework in which 'subsidiary theories' (specific collections of the above statements of possibility) can be placed. In developing this framework, certain properties immediately fall out which are explored first. So far no subsidiary theories have been written down.

%%%%%%%%%%%%%%%%%%%%%%%%%%%%%%%%%%%%
% Give credit where credit is due. %
% Say thanks!                      %
%%%%%%%%%%%%%%%%%%%%%%%%%%%%%%%%%%%%
\section*{Acknowledgements}
These notes are entirely based upon the work of David Deutsch, Chiara Marletto and their collaborators, and are only intended to be a summary of such for purposes of my own understanding.

%%%%%%%%%%%%%%%%
% NEW CHAPTER! %
%%%%%%%%%%%%%%%%
\chapter{Principles and Corollaries of Constructor Theory}

\begin{chapquote}{Richard P. Feynman}
``Natures uses only the longest threads to weave her patterns, so each small piece of her fabric reveals the organization of the entire tapestry.''
\end{chapquote}

\section{Definitions}

\subsection{Concepts}

\begin{itemize}
	\item[\textbf{Constructor Theory}] The framework of principles that regulate how subsidiary theories can be written.
	\item[\textbf{Subsidiary Theory}] A collection of statements of what transformations are and are not possible, written in constructor-theoretic language.
	\item[\textbf{Substrate}] A system that has state and can cause or react to transformations.
	\item[\textbf{Generic Substrate}] Substrates that occur naturally that can be prepared, via tasks, as other (potentially composite) substrates.
	\item[\textbf{Constructor}] A substrate that facilitates a transformation on other substrates with no change to its own state.
	\item[\textbf{Variable}] A set of disjoint attributes for which a substrate can be in a state with some distribution over these attributes. \textit{Note}: two substrates in combination may have variables (and so attributes) that are not present independently in the two substrates - such as distance between centre of masses.
	\item[\textbf{Attribute}] A property of a substrate, capable of being modified by some transformation. An attribute can be defined as the set of states of a substrate that hold that attribute.
	\item[\textbf{Substrate State}] A set of fixings of variables of the substrate in question.
	\item[\textbf{Sharp Variable}] A variable of which a specific attribute is held in a substrate's state. That is to say, if that variable was measured on that substrate, the specific attribute would be evaluated.
	\item[\textbf{Task}] A transformation that is described by a set of input attributes that map in an ordered fashion to a same-size set of output attributes. A task acts on any substrate with at least one attribute in its state that is indistinguishable from an attribute in the input attributes of the task, and changes that attribute to the corresponding output attribute.
	\item[\textbf{Identity Task}] The task that makes no change to the state of a substrate on which it acts.
\end{itemize}

\subsection{Mathematical Syntax}

\begin{itemize}
	\item[\textbf{Substrate}] $\textbf{\textsc{S}}$ is the set of all states the substrate can be in (precise definitions from subsidiary theories). Latin alphabet letters in same font are other substrates, $\textbf{\textsc{E}}$ often being used to denote the environment.
	\item[\textbf{Constructor}] Constructors being substrates use the same font rules as all other substrates, but we reserve the character  $\textbf{\textsc{C}}$ to denote a substrate that is a constructor. Specifically, given a task denoted $\mathcal{T}$ for which a constructor facilitates realisation, we would write that constructor as $\textbf{\textsc{C}}_\mathcal{T}$.
	\item[\textbf{Substrate State}] $\textsc{S} \in \textbf{\textsc{S}}$ is a state of a substrate $\textbf{\textsc{S}}$. Latin alphabet letters in same font are other substrate states, except where specified.
	\item[\textbf{Variable}] $X$ is the set of disjoint attributes that make up the variable (precise definitions from subsidiary theories). Upper-case Latin alphabet letters in the same font are other variables, except where specified.
	\item[\textbf{Attribute}] $x \in X$ is an attribute in the variable $X$. An attribute can also be a member of a substrate state: $x \in \textsc{S}$. Lower-case Latin alphabet characters in the same font are other attributes, except where specified.
	Another useful way of talking about an attribute is as the set of all states of a substrate $\textbf{\textsc{S}}$ containing that attribute: $$\textsc{S}_x = \set{\, \textsc{S} \ ; \ \forall \ \textsc{S} \in \textbf{\textsc{S}}, \ x \in \textsc{S} \,}$$ We use symbols of the kind $\textsc{S}_x$ to refer to such, using the same font as for substrate states with a subscript of the same font as used to indicate attributes.
	\item[\textbf{Task}] $\mathcal{T} = \{\,x_{i} \rightarrow y_{i}\,\}, \ \forall \ i \in \{1, 2, 3, \ldots, N\}$ is a task described by a set of mappings from input attributes to output attributes. In the given definition, $N \in \mathbb{Z}^{+}$ is the number of mappings necessary to define the given task. Latin alphabet characters in the same font are other tasks, except where specified.
	We mark a task as possible by writing $\mathcal{T}^{\underline{\text{\cmark}}}$ and likewise a task as impossible by writing $\mathcal{T}^{\underline{\text{\xmark}}}$.
\end{itemize}

Given the above definitions, we can work with multiple substrates, applying composite tasks that act across them. For example, given the tasks $\mathcal{A}$ and $\mathcal{B}$, and the substrate states $\textsc{M} \in \textbf{\textsc{M}}$ and $\textsc{N} \in \textbf{\textsc{N}}$ we can write a composite task $$\mathcal{A} \otimes \mathcal{B}$$ and a composite substrate state $$\textsc{M} \oplus \textsc{N} \in \textbf{\textsc{M}} \oplus \textbf{\textsc{N}} \ .$$ The tasks can then act as: $$(\mathcal{A} \otimes \mathcal{B})(\textsc{M} \oplus \textsc{N}) = \mathcal{A}\textsc{M} \oplus \mathcal{B}\textsc{N} \ .$$

\subsection{Some Tasks}

\subsubsection{Reversible Computation Task}

For a substrate state $\textsc{S} \in \textbf{\textsc{S}}$ with at least two attributes $x \in \textsc{S}$ and a permutation $\textsc{\Pi}_{\textsc{S}}(x)$, we define a reversible computation task as:
\begin{equation}
	\mathcal{C}_{\Pi}(\textsc{S}) = \underset{x \in \textsc{S}}{\bigcup}\{\,x \rightarrow \textsc{\Pi}_{\textsc{S}}(x)\,\}
\end{equation}This task allows us to define some terms:
\begin{itemize}
	\item[\textbf{Computation Variable}] $\textsc{S}$ is defined as such if ${\mathcal{C}_{\Pi}}^{\underline{\text{\cmark}}} \ \forall \ \textsc{\Pi}_{\textsc{S}}$.
	\item[\textbf{Computation Medium}] as a substrate with at least one computation variable.
\end{itemize}

\subsubsection{Cloning Task}

For a substrate state $\textsc{S} \in \textbf{\textsc{S}}$ with at least two attributes $x \in \textsc{S}$, we can write a cloning task that acts on $ \textbf{\textsc{S}} \oplus \textbf{\textsc{S}}$ as:
\begin{equation}
	\mathcal{R}(\textsc{S}, x_0) = \underset{x \in \textsc{S}}{\bigcup}\{\,(x, x_0) \rightarrow (x, x)\,\}
\end{equation}where each $x_0$ is some attribute on the second substrate which can be transformed into the corresponding attribute $x \in \textsc{S}$. The second substrate, whose state must be one in the set $\textsc{S}_{x_0} \subseteq \textbf{\textsc{S}}, \ x_0 \in \textsc{S} \ \forall \ \textsc{S} \in \textsc{S}_{x_0}$, is required to be derivable from generic, naturally occurring resources.

This cloning task is a generalisation of total cloning, which would require $\textsc{S}$ be the set of all attributes in $\textbf{\textsc{S}}$. As with the reversible computation task we can introduce some definitions:
\begin{itemize}
	\item[\textbf{Clonable State}] $\textsc{S}$ is defined as clonable if $\mathcal{R}(\textsc{S}, x_0)^{\underline{\text{\cmark}}}$.
	\item[\textbf{Clonable Variable}] $X$ is defined as clonable if $\mathcal{R}(X, x_0)^{\underline{\text{\cmark}}}$. (Note that this is okay to write as both a substrate state and a variable are expressed as sets of attributes.)
	\item[\textbf{Information Variable}] $X$ is defined as such if it is a clonable \textit{computation variable}.
	\item[\textbf{Information Attribute}] $x$ is defined as such if $x \in X$, where $X$ is an information variable.
	\item[\textbf{Information Medium}] as a substrate with at least on information variable.
\end{itemize}

It is interesting to note that at this point we have the basis of classical information: given a substrate $\textbf{\textsc{S}}$ with state $\textsc{S} \in \textbf{\textsc{S}}$ has some information variable, $X$ with attributes $x, y \in X$, which is sharp with attribute $x \in S$ and the substrate's state could have been in other configurations where $X$ would have been sharp with attribute $y \in S$, then we have the basic idea of classical information along side the properties of permutations and cloning.

\section{Measurement}

The progress made so far through the definitions obtained above allows us now to consider what conditions would allows us to say any set of attributes are distinguishable from one another. In fact the general form of the task needed to be possible for a set of attributes to be distinguishable is relatively simple:
\begin{equation}
	\mathcal{D}(X) = \underset{x \in X}{\bigcup}\{\,x \rightarrow y_x\,\}
\end{equation}where the set $\underset{x \in X}{\bigcup}\{\,y_x\,\} \ \hat{=} \ Y$ is an information variable.

\begin{itemize}
	\item[\textbf{Distinguishable}] attributes $x$ and $y$ are defined as such if $\mathcal{D}(\{x, y\})^{\underline{\text{\cmark}}}$.
	We use the shorthand $x \perp y$ to represent this property. In the contrary case, we write $x \ \slashed{\perp} \ y$.
\end{itemize}

Distinguishability is essential to performing measurements: after all, fundamentally any measurement, even simply perceiving what we see, is the action of saying what is different about things (one thing is here, another is there; one thing is red, another is blue). That said, for some variable, $X$, to be measurable we need to describe an action by which the original substrate is left intact (though perhaps in a different state). For this, we introduce a second \textit{output} substrate that is prepared to do the measuring - for which it is required to be an information medium. We write the task:
\begin{equation}
	\mathcal{M}(X) = \underset{x \in X}{\bigcup}\{\,(x, \, x_0) \rightarrow (y_x, \,  \text{`}x\text{'})\,\}
\end{equation} where $y_x \in X$, $x_0$ is some initial attribute in the state of the output substrate needed to perform the measurement, and $\text{`}x\text{'}$ is an information attribute that represents the outcome of measuring the attribute $x$ in the state of the original substrate.

Any measurer of variable $X$ on a substrate is also therefore a measurer of any variable $Y \subseteq X$.

We therefore have some new definitions:
\begin{itemize}
	\item[\textbf{Measurable}] attributes ${x} \ \hat{=} \ X$ are defined as such if $\mathcal{M}(X)^{\underline{\text{\cmark}}}$.
	\item[\textbf{Non-perturbing}] measurements are any measurements where $y_x$ as in the definition of $\mathcal{M}(X)$ has the property {\color{red}$y_x \subseteq x$ (I think: $y_x = x \ \forall \ x \in X$)}.
\end{itemize}

\section{Principles}

\subsection{Universal Principles}

The following three principles of constructor theory are assumed to hold universally, that is, for all substrates and tasks.

\begin{itemize}
	\item[\textbf{Counter-factual Principle}] Subsidiary laws are expressed entirely as a collection of statements saying what transformations are and are not possible on system states.
	\item[\textbf{Locality Principle}] A transformation acting on one substrate only affects state on that substrate. That is, for substrates $\textbf{\textsc{S}}_{1}$,  $\textbf{\textsc{S}}_{2}$, with an arbitrary task $\mathcal{A}$, and identity task $\mathcal{I}$:
		\begin{equation}
			\begin{split}
				\forall \ \textsc{S}_{1} \in \textbf{\textsc{S}}_{1}, \ \textsc{S}_{2} \in \textbf{\textsc{S}}_{2} \\
				\textit{state}(\textbf{\textsc{S}}_{1} \oplus \textbf{\textsc{S}}_{2}) = \textsc{S}_{1} \oplus \textsc{S}_{2} \\
				(\mathcal{A} \otimes \mathcal{I})(\textsc{S}_{1} \oplus \textsc{S}_{2}) = \mathcal{A}\textsc{S}_{1} \oplus \ \textsc{S}_{2}
			\end{split}
		\end{equation}
	\item[\textbf{Interoperability Principle}] Given two substrates, $\textbf{\textsc{M}}$ and $\textbf{\textsc{N}}$, with information variables $X$ and $Y$ one apiece, the combined substrate $\textbf{\textsc{M}} \oplus \textbf{\textsc{N}}$ has the information variable: $X \times Y = \set*{(x, y); x \in X, y \in Y}$. Thus combined substrates must allow copying information from one to the other (dark matter couldn't, for example, be truly ``dark").
\end{itemize}

\subsection{Non-Universal Principles}

The following five principles of constructor theory are required to hold for at least some class of substrates, but are assumed to be true universally.

\begin{itemize}
	\item[\textbf{Variable Distinguishability}] For a variable $X$, if each attribute is distinguishable from each other attribute, then we can also say that $X$ is distinguishable from other variables. We can write:
		\begin{equation}
		x \perp y \ \forall \ x,\, y \in X, \ x \neq y \ \Rightarrow \ X \perp Y \ \forall \ Y, \ X \neq Y
		\end{equation}where $Y$ are also variables.
	\item[\textbf{State Distinguishability}] For some attributes $x$ and $y$, if all states holding $x$ are distinguishable from all states holding $y$, then we can conclude that $x$ and $y$ are also distinguishable. We can write:
		\begin{equation}
			\textsc{L} \perp \textsc{M} \ \forall \ \textsc{L} \in \textsc{S}_x, \ \forall \ \textsc{M} \in \textsc{S}_y \ \Rightarrow \ x \perp y
		\end{equation} where $\textsc{S}_x$ and $\textsc{S}_y$ are the sets of all states containing the attribute $x$ and $y$ respectively. We also permit the language ``attribute $x$ is perpendicular to states, $\textsc{S}_y$ of attribute $y$", which can be written $x \perp \textsc{S}_y$, which is an equivalent statement as the left-hand side of (1.7).
	\item[\textbf{Infinite Resources}] An infinite number of instances of any information medium can be prepared from generic substrates. This lets us state:
		\begin{equation}
			\mathcal{T}^{\underline{\text{\cmark}}} \ \Rightarrow \ \{\textsc{g} \rightarrow \textbf{\textsc{C}}_\mathcal{T}\}^{\underline{\text{\cmark}}}
		\end{equation}which is to say that for a task to be possible the corresponding constructor must be preparable from generic substrates. That a constructor for a possible task must be preparable from generic substrates in turn tells us that there exist some set of generic constructors that naturally occur - elsewise we'd have no bootstrapping mechanism to get to more interesting tasks - which means:
		\begin{equation}
			\mathcal{T}^{\underline{\text{\cmark}}} \ \Leftrightarrow \ (\exists\textsc{h})(\mathcal{T} \times \{\textsc{g} \rightarrow \textsc{h}\})^{\underline{\text{\cmark}}} \ \text{.}
		\end{equation}
	\item[\textbf{Arbitrary Complexity}] A regular network created out of an arbitrary number of possible tasks itself is a possible task.
\end{itemize}

Variable distinguishability is not logically required by the definition of variables, and likewise state distinguishability is not logically required by the definitions of attributes and states, but these assumptions seem like a useful simplification when dealing with reality where we expect regularity in observable phenomena (in this case distinguishability of observed quantities) to have a unifying explanation. That said, if it turns out these do not hold, we can weaken the assumptions by making them non-universal. Together, the infinite resources and arbitrary complexity principles look very similar to the assumption of an infinite tape in the Turing machine model, this is of course an idealisation and in experience we see that there are strict limitations of resource and complexity possibilities.

\section{Concepts}

Given the constructor-theoretic way of finding classical information as discussed in definitions of the cloning task, we realise from the \textit{locality principle} that the information capacity of disjoint substrates is the sum of each of their capacities - as one would expect. This definition of classical information also has the nice benefit of being completely removed from any physical laws and lacks any logical circularities that arise in existing descriptions of classical information.

\section{Philosophical Discussion}

\end{document}
